\documentclass[12pt, a4paper]{report}
\setcounter{secnumdepth}{3}

\usepackage[a4paper, total={6in, 10in}]{geometry}

\usepackage[portuges]{babel}
\usepackage[utf8]{inputenc}
\usepackage{graphicx}
\usepackage{url}
\usepackage{enumerate}
\usepackage{xspace}


% Documento
\begin{document}

\title{
    Fundamentos de Sistemas Distribuídos\\
    \textbf{\\Trabalho Prático}
    \large{\\Relatório de Desenvolvimento}
}

\author{
    Miguel Oliveira\\ pg41088
    \and Pedro Moura\\ pg41094
    \and César Silva\\ pg41842
}
\date{Universidade do Minho,\\\today}

\maketitle

\tableofcontents

\chapter{Introdução}
O seguinte relatório descreve o desenvolvimento do trabalho prático da UC de Paradigmas de Sistemas Distribuídos.

O trabalho prático tem como objetivo aplicar os conhecimentos adquiridos nas aulas, mais nomeadamente protocolos de comunicação entre diferentes  (Java e Erlang)linguagens utilizando serviços de serialização como \textit{Protocol Buffers}, programação com atores, \textit{Message Oriented Middleware} e serviços REST.


\chapter{Cliente}

\chapter{Front-End}

O comportamento inicial do Front-End começa pela criação de um
\textit{ListeningSocket} para aceitar ligações e um actor especial que 
denominamos \textit{Login Manager} que comunica via mensagens nativas
\textit{Erlang}, cuja função é validar credenciais recebidas. \\
Quando é recebida uma ligação, é alocado um ator para se encarregar desta.
Este pode ser dividido em 2 grandes fases - autenticado e não autenticado.
Tudo o que um utilizador não autenticado pode fazer é realizar tentativas de
autenticação. Por sua vez quando um utilizador é autenticado, este abre uma
ligação a um \textbf{Negociador} e a partir deste ponto, simplesmente encaminha
pedidos vindos do cliente para este. A comunicação usada entre
o \textbf{Cliente} e o \textbf{Front-End} usa \textit{gpb}, uma implementação em
\textit{Erlang} de \textit{Protobuffers}
- \url{https://github.com/tomas-abrahamsson/gpb} - o qual apresenta um guia
extenso em como usar o módulo.

 
\chapter{Negociador}

\chapter{Catálogo}

\chapter{Conclusão}

Achamos que a solução proposta, responde aos vários problemas propostos no
enunciado, quer a nível de uso de conceitos usados - \textit{ZeroMQ},
\textit{REST}, \textit{Erlang} - como também ao nível das funcionalidades
pedidas.
Como trabalho futuro, poderiamos considerar a adesão de negociadores em
\textit{run-time}, tópicos de subscrição mais extensos ou até uma interface
gráfica para o cliente, de maneira a consolidar ainda mais o implementado.
\end{document}
