\documentclass[12pt, a4paper]{report}
\setcounter{secnumdepth}{3}

\usepackage[a4paper, total={6in, 10in}]{geometry}

\usepackage[portuges]{babel}
\usepackage[utf8]{inputenc}
\usepackage{graphicx}
\usepackage{url}
\usepackage{enumerate}
\usepackage{xspace}


% Documento
\begin{document}

\title{
    Fundamentos de Sistemas Distribuídos\\
    \textbf{\\Trabalho Prático}
    \large{\\Relatório de Desenvolvimento}
}

\author{
    Miguel Oliveira\\ pg41088
    \and Pedro Moura\\ pg41094
    \and César Silva\\ pg41842
}
\date{Universidade do Minho,\\\today}

\maketitle

\tableofcontents

\chapter{Introdução}
O seguinte relatório descreve o desenvolvimento do trabalho prático da UC de Paradigmas de Sistemas Distribuídos.

O trabalho prático tem como objetivo aplicar os conhecimentos adquiridos nas aulas, mais nomeadamente protocolos de comunicação entre diferentes  (Java e Erlang)linguagens utilizando serviços de serialização como \textit{Protocol Buffers}, programação com atores, \textit{Message Oriented Middleware} e serviços REST.


\chapter{Cliente}

\chapter{Front-End}

Na implementa

\chapter{Negociador}

\chapter{Catálogo}

\end{document}
